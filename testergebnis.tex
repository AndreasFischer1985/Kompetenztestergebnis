\documentclass[12pt]{article}
\usepackage{ucs}
\usepackage[utf8x]{inputenc}
\usepackage[T1]{fontenc}
\usepackage[ngerman]{babel}
\usepackage{multicol}
\usepackage[absolute]{textpos}
\RequirePackage{tikz}
\usepackage{fontawesome}
\setlength{\parindent}{0pt} % Disable paragraph indentation
\definecolor{customcolor}{rgb}{0,0.3294118,0.4784314}
\newcommand{\e}{\textcolor{gray}{\faCircleO}}
\newcommand{\f}{\textcolor{customcolor}{\faCircle}}
\newcommand{\0}{\e\e\e\e\e}
\newcommand{\1}{\f\e\e\e\e}
\newcommand{\2}{\f\f\e\e\e}
\newcommand{\3}{\f\f\f\e\e}
\newcommand{\4}{\f\f\f\f\e}
\newcommand{\5}{\f\f\f\f\f}
\newcommand{\nextcolumn}{\vfill\columnbreak}
\newcommand{\myMargin}{5cm}
\usepackage[left=\myMargin, right=1cm, top=0.5cm, bottom=0.5cm]{geometry}
\newcommand{\logo}{\includegraphics[scale=0.25]{logo.png}}
\newcommand{\score}[3]{
	\hrule
	\vspace{0.4cm}
	\large\textbf{#1}\newline
	\small	
	%\vspace{0.5cm}
	%\begin{multicols}{2}	
	\begin{minipage}[c]{0.7\linewidth}
	#2
	\end{minipage}
	%\nextcolumn
	\begin{minipage}[c]{0.27\linewidth}
	\rightline{\large{#3}}
	\end{minipage}
	\vspace{0.4cm}
	%\end{multicols}
	%\vspace{0.1cm}
	\hrule
}
\newcommand{\information}[4]{
	\hrule 
	\vspace{0.5cm}	
	\textbf{Name:} \small{#1}\begin{multicols}{3}
	\textbf{Geburtsdatum:} \small{#2}
	\nextcolumn
	\textbf{Testdatum:} \small{#3}
	\nextcolumn
	\textbf{Testsprache:} \small{#4}
	\end{multicols}	
	\vspace{0.1cm}\hrule
}
\newcommand{\globalframe}{
	\begin{textblock}{2}(0.5, 0.5)
	\begin{tikzpicture}[remember picture,overlay]
 		\node [rectangle,fill=customcolor,minimum width=130,minimum height=\paperheight*2](box) at (1.5,0){};
	\end{tikzpicture}
	Entwurfsfassung
	\end{textblock}
}
\begin{document}
	\huge\textbf{Testergebnis} \hspace{5.5cm}\logo
	\newline\large zur Kompetenz im Bereich xy
	\normalsize
	\vspace{0.5cm}	
	\globalframe
	\information{Dr. Andreas Fischer}{21.08.1985}{02.02.2019}{deutsch}
	\vspace{0.5cm}
	\small Der durchgeführte Test umfasst 6 ausweisbare Teilergebnisse. Jedes Teilergebnis basiert auf mehreren Antworten zu unterschiedlichen Inhalten des jeweiligen Gegenstandsbereichs.
	\vspace{0.5cm}
  	%\begin{multicols}{2} %	\nextcolumn
  	\score{Teilergebnis 1}{Beschreibung der Testinhalte in beliebiger Ausführlichkeit. Maximal drei Zeilen sind zu empfehlen, da andernfalls die Ergebnisse nicht mehr auf eine Seite zu passen drohen.}{\1}
  	\score{Teilergebnis 2}{Beschreibung der Testinhalte in beliebiger Ausführlichkeit. Maximal drei Zeilen sind zu empfehlen, da andernfalls die Ergebnisse nicht mehr auf eine Seite zu passen drohen.}{\2}
	\score{Teilergebnis 3}{Beschreibung der Testinhalte in beliebiger Ausführlichkeit. Maximal drei Zeilen sind zu empfehlen, da andernfalls die Ergebnisse nicht mehr auf eine Seite zu passen drohen.}{\3}
	\score{Teilergebnis 4}{Beschreibung der Testinhalte in beliebiger Ausführlichkeit. Maximal drei Zeilen sind zu empfehlen, da andernfalls die Ergebnisse nicht mehr auf eine Seite zu passen drohen.}{\4}
	\score{Teilergebnis 5}{Beschreibung der Testinhalte in beliebiger Ausführlichkeit. Maximal drei Zeilen sind zu empfehlen, da andernfalls die Ergebnisse nicht mehr auf eine Seite zu passen drohen.}{\5}
	\score{Teilergebnis 6}{Beschreibung der Testinhalte in beliebiger Ausführlichkeit. Maximal drei Zeilen sind zu empfehlen, da andernfalls die Ergebnisse nicht mehr auf eine Seite zu passen drohen.}{\0}
	\vspace{0.6cm}
	\scriptsize
	\textbf{Legende:}\newline
	\0 = keine Aussage möglich\newline
	\1 = geringe Kompetenz\newline
	\2 = grundlegende Kompetenz\newline
	\3 = mittlere Kompetenz\newline
	\4 = hohe Kompetenz\newline
	\5 = außergewöhnlich hohe Kompetenz\newline
	\normalsize
\end{document}